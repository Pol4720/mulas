\documentclass[10pt,aspectratio=169]{beamer}

% Paquetes esenciales
\usepackage[utf8]{inputenc}
\usepackage[spanish]{babel}
\usepackage{amsmath,amsfonts,amssymb,amsthm}
\usepackage{graphicx}
\usepackage{tikz}
\usepackage{xcolor}
\usepackage{booktabs}

% Tema y colores
\usetheme{Madrid}
\usecolortheme{default}

% Colores personalizados
\definecolor{CustomBlue}{RGB}{0, 102, 204}
\definecolor{CustomGreen}{RGB}{0, 153, 76}

\setbeamercolor{structure}{fg=CustomBlue}

% Título y autor
\title{Problema de Transporte Logístico Discreto}
\subtitle{Diseño y Análisis de Algoritmos}
\author{Richard A. Matos Arderí \\ Abel Ponce González \\ Abraham Romero Imbert}
\institute{Facultad de Matemática y Computación}
\date{\today}

% Macros útiles
\newcommand{\highlight}[1]{\textcolor{CustomBlue}{\textbf{#1}}}

\begin{document}

%=== PORTADA ===
\frame{\titlepage}

%=== ÍNDICE ===
\begin{frame}
\frametitle{Contenidos}
\tableofcontents[hideallsubsections]
\end{frame}

%=== SECCIÓN 1: PROBLEMA ===
\section{El Problema}

\begin{frame}
\frametitle{Descripción del Problema}

\textbf{Escenario Real:}
\begin{itemize}
    \item \highlight{n} paquetes con peso y valor
    \item \highlight{k} vehículos con capacidad máxima
    \item Distribuir respetando límites de capacidad
    \item \textbf{Objetivo:} Equilibrar la carga
\end{itemize}

\vspace{0.5cm}
\textbf{Ejemplo Simple:}
\begin{itemize}
    \item 4 paquetes: 2kg, 3kg, 4kg, 1kg
    \item 2 camiones: 6kg uno y 5kg otro
    \item Desafío: distribución equitativa
\end{itemize}

\vspace{0.7cm}
\centering
\textbf{[ESPACIO PARA ILUSTRACIÓN: Camiones y paquetes]}

\end{frame}

%=== SECCIÓN 2: FORMALIZACIÓN ===
\section{Formalización Matemática}

\begin{frame}
\frametitle{Definiciones Fundamentales}

\textbf{Entrada del Problema:}
\begin{itemize}
    \item Ítems: Un ítem $i \in I$ se caracteriza por un par $(w_i, v_i)$, $I = \{1, 2, \ldots, n\}$
    \item Pesos: $w_i > 0$ para cada ítem $i$
    \item Valores: $v_i \geq 0$ para cada ítem $i$
    \item $k$ contenedores con capacidades $C_1, \ldots, C_k$
\end{itemize}

\vspace{0.3cm}
\textbf{Variables de Decisión:}
\[
x_{ij} = \begin{cases} 1 & \text{si ítem } i \text{ va a contenedor } j \\ 0 & \text{en otro caso} \end{cases}
\]

\vspace{0.3cm}
\textbf{Objetivo:}
$\text{Minimizar: } \max_j V_j - \min_j V_j $ \\
donde $V_j = \sum_{i=1}^{n} v_i \cdot x_{ij}$

\textbf{Sujeto a:} 
$\forall j \in \{1, \ldots, k\}: \sum_{i: \sigma(i) = j} w_i \leq C_j$


\end{frame}

\begin{frame}
\frametitle{Formulación como Programa Lineal Entero}

\textbf{Planteo ILP:}

\begin{align*}
\text{minimizar} \quad & z^+ - z^- \\
\text{s.a.:} \quad & \sum_{j=1}^{k} x_{ij} = 1, \quad \forall i \\
& \sum_{i=1}^{n} w_i \cdot x_{ij} \leq C_j, \quad \forall j \\
& \sum_{i=1}^{n} v_i \cdot x_{ij} \leq z^+, \quad \forall j \\
& \sum_{i=1}^{n} v_i \cdot x_{ij} \geq z^-, \quad \forall j \\
& x_{ij} \in \{0, 1\}
\end{align*}

\textbf{Propiedad:} En optimalidad: $z^+ = \max_j V_j$ y $z^- = \min_j V_j$

\end{frame}

%=== SECCIÓN 3: COMPLEJIDAD ===
\section{Análisis de Complejidad}

\begin{frame}
\frametitle{Problema de Decisión vs Optimización}

\textbf{Problema de Decisión (BALANCED-BIN-PACKING-DEC):}
\begin{quote}
Dados $n$ ítems con pesos y valores, $k$ bins con capacidades $C_1, \ldots, C_k$, y un umbral $B$, ¿existe una asignación factible tal que la diferencia máxima de valores entre bins sea $\leq B$?
\end{quote}

\vspace{0.5cm}
\textbf{Problema de Optimización (BALANCED-BIN-PACKING-OPT):} 
\begin{quote}
Dados $n$ ítems con pesos y valores, $k$ bins con capacidades $C_1, \ldots, C_k$, encontrar una asignación factible que minimice la diferencia máxima de valores entre bins.
\end{quote}

\vspace{0.5cm}
\highlight{Proposición:} Si el problema de decisión BALANCED-BIN-PACKING-DEC está en NP, entonces el problema de optimización BALANCED-BIN-PACKING-OPT está en NPO (problemas de optimización NP).

\end{frame}

\begin{frame}
    \textbf{Paso 1: BALANCED-BIN-PACKING-DEC $\in$ NP}

Un certificado para una instancia con respuesta 'sí' es una asignación $\sigma: I \rightarrow \{1, \ldots, k\}$. La verificación requiere:
\begin{enumerate}
    \item Verificar que cada ítem está asignado: $O(n)$
    \item Calcular peso total de cada bin: $O(n)$
    \item Verificar restricciones de capacidad: $O(k)$
    \item Calcular valor total de cada bin: $O(n)$
    \item Verificar que $\max_j V_j - \min_j V_j \leq B$: $O(k)$
\end{enumerate}

Total: $O(n + k)$, por lo tanto el certificado es verificable en tiempo polinomial. $\square$

\end{frame}

\begin{frame}
\frametitle{Cadena de Reducciones}

\centering
\Large
\vspace{1cm}
PARTITION $\leq_p$ 3-PARTITION $\leq_p$ BALANCED-BIN-PACKING

\vspace{1cm}
\normalsize
\textbf{Implicación:} El problema es \textbf{NP-Hard}

\vspace{0.5cm}

\textbf{Resultado:} El problema es NP-Completo

\vspace{0.5cm}
No existe algoritmo polinomial conocido (asumiendo P $\neq$ NP)

\vspace{0.5cm}
Se requiere trade-off: \highlight{optimalidad vs velocidad}

\end{frame}

%=== SECCIÓN 4: MÉTODOS ===
\section{Métodos de Resolución}

\begin{frame}
\frametitle{Métodos Implementados (por tipo)}

\textbf{Exactos (garantizan óptimo):}
\begin{itemize}
    \item Fuerza Bruta
    \item Branch \& Bound
    \item Programación Dinámica
\end{itemize}

\textbf{Greedy / Aproximados (muy rápidos):}
\begin{itemize}
    \item FFD (First Fit Decreasing)
    \item BFD (Best Fit Decreasing)
    \item WFD (Worst Fit Decreasing)
    \item LDF (Largest Difference First)
    \item LPT Balanced (Longest Processing Time)
    \item KK (Karmarkar-Karp) para particionamiento
    \item Round Robin balanceado
\end{itemize}

\textbf{Metaheurísticas (búsqueda inteligente):}
\begin{itemize}
    \item Simulated Annealing
    \item Algoritmos Genéticos
    \item Búsqueda Tabú
\end{itemize}

\vspace{0.2cm}
\scriptsize
\highlight{Nota:} Greedy prioriza velocidad; metaheurísticas balancean calidad/tiempo; exactos garantizan optimalidad en instancias pequeñas.

\end{frame}

\begin{frame}
\frametitle{Búsqueda Exhaustiva (Fuerza Bruta)}

\textbf{Estrategia:} Enumerar y evaluar todas las $k^n$ asignaciones posibles.

\vspace{0.3cm}
\textbf{Garantías:} \highlight{Encuentra el óptimo global garantizado}

\vspace{0.5cm}
\textbf{Límites prácticos (1 segundo):}
\begin{itemize}
    \item $k=2$: hasta $n=14$ (16K asignaciones)
    \item $k=3$: hasta $n=11$ (177K asignaciones)
    \item $k=4$: hasta $n=8$ (65K asignaciones)
\end{itemize}

\vspace{0.5cm}
\textbf{Complejidad:}
\begin{itemize}
    \item \textbf{Tiempo:} $O(k^n \cdot n)$ - Exponencial en ambas dimensiones
    \item \textbf{Espacio:} $O(n + k)$ - Muy eficiente
\end{itemize}

\vspace{0.3cm}
\textbf{Aplicabilidad:} Validar heurísticas, instancias pequeñas ($n \leq 14$)

\end{frame}

\begin{frame}
\frametitle{Branch and Bound - Búsqueda con Poda}

\textbf{Idea Central:} Fuerza Bruta + Cotas Inferiores = Poda Inteligente

\vspace{0.3cm}
\textbf{Algoritmo:}
\begin{enumerate}
    \item Construir árbol de decisión rama por rama
    \item Para cada nodo: calcular \textbf{cota inferior} optimista
    \item Si cota $\geq$ mejor solución encontrada: \highlight{podar rama}
    \item Continuar con ramas prometedoras
\end{enumerate}

\vspace{0.3cm}
\textbf{Mejora respecto Fuerza Bruta:}
\begin{itemize}
    \item Explora \highlight{solo subárbol prometedor}
    \item Cotas ajustadas = mayor poda
    \item Encontrar óptimo rápido = mejor umbral de poda
\end{itemize}

\vspace{0.3cm}
\textbf{Complejidad:} $O(k^n)$ peor caso, pero mucho mejor en práctica

\vspace{0.3cm}
\textbf{Garantías:} \highlight{Optimalidad + velocidad} en instancias medianas

\end{frame}

\begin{frame}
\frametitle{Programación Dinámica - Construcción Óptima}

\textbf{Estrategia:} Construir solución de forma incremental: contenedor por contenedor.

\vspace{0.3cm}
\textbf{Subestructura Óptima:} Si tenemos la mejor asignación a $j-1$ contenedores, podemos encontrar la mejor para $j$ contenedores añadiendo uno nuevo.

\vspace{0.3cm}
\textbf{Estado DP:} $DP[j][mask]$ = mejor configuración de ítems en $mask$ usando primeros $j$ contenedores

\begin{itemize}
    \item $j$: número de contenedores (1 a $k$)
    \item $mask$: subconjunto binario de ítems
\end{itemize}

\vspace{0.3cm}
\textbf{Recurrencia:} Para cada contenedor $j$, probar todos los subconjuntos de ítems que le caben

\vspace{0.3cm}
\textbf{Complejidad:}
\begin{itemize}
    \item \textbf{Tiempo:} $O(k^2 \cdot 3^n)$ - Factor 3 por subconjuntos
    \item \textbf{Espacio:} $O(k \cdot 2^n)$ - Tabla completa
\end{itemize}

\vspace{0.2cm}
\textbf{Límites:} Instancias pequeñas ($n \leq 15$)

\end{frame}

\begin{frame}
\frametitle{Algoritmos Greedy - Primera Parte}
\footnotesize

\textbf{Estrategia común:} Ordena ítems y asigna cada uno usando una regla local.

\textbf{First Fit Decreasing (FFD):}
\begin{itemize}
    \item Ordena por peso decreciente
    \item Asigna al \textbf{primer contenedor} con espacio disponible
    \item Tiempo: $O(n \log n + n \cdot k)$ | Gap: 5-15\%
\end{itemize}

\textbf{Best Fit Decreasing (BFD):}
\begin{itemize}
    \item Ordena por valor decreciente
    \item Asigna al contenedor que \textbf{minimiza diferencia} resultante
    \item Tiempo: $O(n \log n + n \cdot k)$ | Gap: 4-12\%
\end{itemize}

\textbf{Worst Fit Decreasing (WFD):}
\begin{itemize}
    \item Ordena por valor decreciente
    \item Asigna al contenedor con \textbf{mayor espacio libre}
    \item Distribución uniforme | Tiempo: $O(n \log n + n \log k)$ | Gap: 5-15\%
\end{itemize}

\textbf{LPT Balanced:}
\begin{itemize}
    \item \textbf{Longest Processing Time} para balance
    \item Asigna al contenedor con \textbf{menor valor} actual
    \item Muy rápido | Tiempo: $O(n \log n + n \log k)$ | Gap: 3-10\%
\end{itemize}

\end{frame}

\begin{frame}
\frametitle{Algoritmos Greedy y Aproximación - Segunda Parte}
\footnotesize

\textbf{Round Robin Greedy:}
\begin{itemize}
    \item Distribución secuencial por \textbf{mínimo valor actual}
    \item Tiempo: $O(n \log n + n \log k)$ | Gap: 4-14\%
\end{itemize}

\textbf{Largest Difference First (LDF):}
\begin{itemize}
    \item En cada paso, \textbf{evalúa todas combinaciones} (ítem, contenedor)
    \item Elige la que \textbf{minimiza diferencia} resultante
    \item Más lento pero mejor balance
    \item Tiempo: $O(n^2 \cdot k)$ | Gap: 2-8\%
\end{itemize}

\textbf{Karmarkar-Karp (KK) - Particionamiento:}
\begin{itemize}
    \item Algoritmo clásico de \textbf{diferenciación}
    \item Combina valores: reemplaza dos máximos por su diferencia
    \item Excelente para 2 contenedores
    \item Tiempo: $O(n \log n)$ | Gap: 2-6\%
\end{itemize}

\textbf{Resumen Greedy:}
\begin{itemize}
    \item \highlight{Ventaja:} Muy rápidos, $O(n \log n)$
    \item \highlight{Desventaja:} Gap 2-15\%, sin garantías teóricas
    \item \highlight{Uso:} Baseline rápido, problemas grandes
\end{itemize}

\end{frame}

\begin{frame}
\frametitle{Metaheurísticas}

Simulated Annealing, Algoritmos Genéticos, Búsqueda Tabú

\vspace{0.5cm}
\textbf{Esquema general:}
\begin{enumerate}
    \item Generar solución inicial
    \item Mejorar iterativamente con movimientos locales
    \item Aceptar movimientos malos ocasionalmente
    \item Detener por convergencia o tiempo
\end{enumerate}

\vspace{0.5cm}
\textbf{Ventajas:} Rápidas, buenos resultados

\textbf{Desventajas:} Sin garantías, aleatoriedad, parámetros

\end{frame}

\begin{frame}
\frametitle{Análisis de Complejidad - Comparativa}

\small
\begin{center}
\textbf{Complejidad temporal y espacial de los algoritmos implementados}
\end{center}

\vspace{0.1cm}
\small
\begin{center}
\begin{tabular}{lcc}
\toprule
\textbf{Algoritmo} & \textbf{Tiempo} & \textbf{Espacio} \\
\midrule
Fuerza Bruta & $O(k^n \cdot n)$ & $O(n + k)$ \\
FFD & $O(n \log n + n \cdot k)$ & $O(n + k)$ \\
BFD & $O(n \log n + n \cdot k)$ & $O(n + k)$ \\
WFD & $O(n \log n + n \log k)$ & $O(n + k)$ \\
LPT Balanced & $O(n \log n + n \log k)$ & $O(n + k)$ \\
Round Robin & $O(n \log n + n \log k)$ & $O(n + k)$ \\
LDF & $O(n^2 \cdot k)$ & $O(n + k)$ \\
KK (Karmarkar-Karp) & $O(n \log n)$ & $O(n)$ \\
Prog. Dinámica & $O(k^2 \cdot 3^n)$ & $O(k \cdot 2^n)$ \\
Branch \& Bound & $O(k^n)$ peor caso & $O(n \cdot k)$ \\
Simulated Annealing & $O(I \cdot n)$ & $O(n)$ \\
Genetic Algorithm & $O(G \cdot P \cdot n)$ & $O(P \cdot n)$ \\
Tabu Search & $O(I \cdot N)$ & $O(n + T)$ \\
\bottomrule
\end{tabular}
\end{center}

\vspace{0.3cm}
\small
\textbf{Leyenda:} $n$=ítems, $k$=contenedores, $I$=iteraciones, $G$=generaciones, $P$=población, $N$=vecindario, $T$=lista tabú

\end{frame}

%=== SECCIÓN 5: ANÁLISIS ===
\section{Análisis Experimental}

\begin{frame}
\frametitle{Comparación de Tiempos}

\centering
\includegraphics[width=0.95\textwidth]{images/runtime_vs_size_k3.png}

\small
Instancias aleatorias con $n \in [5, 50]$ e $k=3$

Eje Y: tiempo (segundos, escala logarítmica)

Eje X: número de ítems

\end{frame}

\begin{frame}
\frametitle{Órdenes de Complejidad}

\centering
\includegraphics[width=0.95\textwidth,height= 0.80\textheight]{images/complexity_orders_logarithmic.png}

\scriptsize
Comparación teórica de órdenes de complejidad con $k=3$ contenedores

% - Rojo/Morado: Algoritmos exponenciales (Fuerza Bruta, B\&B, DP)

% - Azul/Verde: Algoritmos polinomiales (Greedy, KK)

% - Naranja/Rosa: Metaheurísticas (lineales con parámetro fijo)

\end{frame}

\begin{frame}
\frametitle{Calidad de Soluciones}

\textbf{Gap de Optimalidad:}
\[
\text{Gap}(\%) = \frac{\text{Heurístico} - \text{Óptimo}}{\text{Óptimo}} \times 100
\]

\vspace{0.3cm}
\centering
\includegraphics[width=0.85\textwidth]{images/gap_by_algorithm.png}

\vspace{0.2cm}
\small
\textbf{Resultado:} Metaheurísticas (SA, GA, Tabú) presentan los gaps más bajos

\end{frame}

\begin{frame}
\frametitle{Resumen Comparativo}

\centering
\footnotesize
\begin{tabular}{|l|c|c|c|l|}
\toprule
\textbf{Algoritmo} & \textbf{Óptimo} & \textbf{Velocidad} & \textbf{Rango} & \textbf{Aplicación} \\
\midrule
\multicolumn{5}{|l|}{\textbf{Métodos Exactos (Garantizan optimalidad)}} \\
\midrule
Fuerza Bruta & Sí & Muy lenta & $n < 15$ & Validación, instancias mínimas \\
Branch \& Bound & Sí & Media & $n < 25$ & Instancias medianas, mejor relación \\
Prog. Dinámica & Sí & Media & $n < 20$ & Instancias pequeñas con subproblemas \\
\midrule
\multicolumn{5}{|l|}{\textbf{Métodos Aproximados/Greedy (Rápidos, sin garantía)}} \\
\midrule
FFD (First Fit Dec.) & No & Muy rápida & Todos & Baseline rápido, gap 5-15\% \\
BFD (Best Fit Dec.) & No & Muy rápida & Todos & Equilibrio simple, gap 4-12\% \\
WFD (Worst Fit Dec.) & No & Muy rápida & Todos & Distribución balanceada, gap 5-15\% \\
LDF (Largest Diff. First) & No & Rápida & Todos & Balance óptimo, gap 2-8\% \\
LPT Balanced & No & Muy rápida & Todos & Greedy mejorado, gap 3-10\% \\
KK (Karmarkar-Karp) & No & Muy rápida & Todos & Particionamiento, gap 2-6\% \\
Round Robin & No & Muy rápida & Todos & Distribución uniforme, gap 4-14\% \\
\midrule
\multicolumn{5}{|l|}{\textbf{Metaheurísticas (Búsqueda inteligente, mejor calidad)}} \\
\midrule
Simulated Annealing & No & Rápida & Todos & Escape de mínimos locales, gap 0.5-2\% \\
Algoritmos Genéticos & No & Rápida & Todos & Problemas complejos, gap 0.5-3\% \\
Búsqueda Tabú & No & Rápida & Todos & Mejor calidad promedio, gap 0.2-1.5\% \\
\bottomrule
\end{tabular}

\end{frame}

%=== CONCLUSIONES ===
\section{Conclusiones}

\begin{frame}
\frametitle{Conclusiones}

\highlight{1. NP-Hard:} No existe algoritmo polinomial (bajo P $\neq$ NP)

\vspace{0.3cm}
\highlight{2. Trade-off:} Optimalidad vs velocidad

\vspace{0.3cm}
\textbf{Elección depende del contexto:}
\begin{itemize}
    \item Instancias pequeñas: B\&B o DP
    \item Problemas reales: Metaheurísticas
    \item Baseline rápido: Greedy (FFD, LPT, KK, LDF)
\end{itemize}

\vspace{0.3cm}
\highlight{3. Herramientas:} 14 algoritmos, dashboard interactivo, benchmarking

\vspace{0.3cm}
\highlight{4. Aplicaciones:} Logística, distribución de carga, computación distribuida

\end{frame}

%=== AGRADECIMIENTOS ===
\begin{frame}
\frametitle{Agradecimientos}

\vspace{2cm}
\centering
\Large
\highlight{Gracias por su atención}

\vspace{1.5cm}
\normalsize
\textbf{Preguntas y Discusión}

\vspace{1cm}
\small
Código disponible en GitHub

\end{frame}

\end{document}
