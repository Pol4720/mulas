\documentclass[10pt,aspectratio=169]{beamer}

% Paquetes esenciales
\usepackage[utf8]{inputenc}
\usepackage[spanish]{babel}
\usepackage{amsmath,amsfonts,amssymb,amsthm}
\usepackage{graphicx}
\usepackage{tikz}
\usepackage{xcolor}
\usepackage{booktabs}

% Tema y colores
\usetheme{Madrid}
\usecolortheme{default}

% Colores personalizados
\definecolor{CustomBlue}{RGB}{0, 102, 204}
\definecolor{CustomGreen}{RGB}{0, 153, 76}

\setbeamercolor{structure}{fg=CustomBlue}

% Título y autor
\title{Problema de Transporte Logístico Discreto}
\subtitle{Diseño y Análisis de Algoritmos}
\author{Richard A. Matos Arderí \\ Abel Ponce González \\ Abraham Romero Imbert}
\institute{Facultad de Matemática y Computación}
\date{\today}

% Macros útiles
\newcommand{\highlight}[1]{\textcolor{CustomBlue}{\textbf{#1}}}

\begin{document}

%=== PORTADA ===
\frame{\titlepage}

%=== ÍNDICE ===
\begin{frame}
\frametitle{Contenidos}
\tableofcontents[hideallsubsections]
\end{frame}

%=== SECCIÓN 1: PROBLEMA ===
\section{El Problema}

\begin{frame}
\frametitle{Descripción del Problema}

\textbf{Escenario Real:}
\begin{itemize}
    \item \highlight{n} paquetes con peso y valor
    \item \highlight{k} vehículos con capacidad máxima
    \item Distribuir respetando límites de capacidad
    \item \textbf{Objetivo:} Equilibrar la carga
\end{itemize}

\vspace{0.5cm}
\textbf{Ejemplo Simple:}
\begin{itemize}
    \item 4 paquetes: 2kg, 3kg, 4kg, 1kg
    \item 2 camiones: 6kg uno y 5kg otro
    \item Desafío: distribución equitativa
\end{itemize}

\vspace{0.7cm}
\centering
\textbf{[ESPACIO PARA ILUSTRACIÓN: Camiones y paquetes]}

\end{frame}

%=== SECCIÓN 2: FORMALIZACIÓN ===
\section{Formalización Matemática}

\begin{frame}
\frametitle{Definiciones Fundamentales}

\textbf{Entrada del Problema:}
\begin{itemize}
    \item Ítems: Un ítem $i \in I$ se caracteriza por un par $(w_i, v_i)$, $I = \{1, 2, \ldots, n\}$
    \item Pesos: $w_i > 0$ para cada ítem $i$
    \item Valores: $v_i \geq 0$ para cada ítem $i$
    \item $k$ contenedores con capacidades $C_1, \ldots, C_k$
\end{itemize}

\vspace{0.3cm}
\textbf{Variables de Decisión:}
\[
x_{ij} = \begin{cases} 1 & \text{si ítem } i \text{ va a contenedor } j \\ 0 & \text{en otro caso} \end{cases}
\]

\vspace{0.3cm}
\textbf{Objetivo:}
$\text{Minimizar: } \max_j V_j - \min_j V_j $ \\
donde $V_j = \sum_{i=1}^{n} v_i \cdot x_{ij}$

\textbf{Sujeto a:} 
$\forall j \in \{1, \ldots, k\}: \sum_{i: \sigma(i) = j} w_i \leq C_j$


\end{frame}

\begin{frame}
\frametitle{Formulación como Programa Lineal Entero}

\textbf{Planteo ILP:}

\begin{align*}
\text{minimizar} \quad & z^+ - z^- \\
\text{s.a.:} \quad & \sum_{j=1}^{k} x_{ij} = 1, \quad \forall i \\
& \sum_{i=1}^{n} w_i \cdot x_{ij} \leq C_j, \quad \forall j \\
& \sum_{i=1}^{n} v_i \cdot x_{ij} \leq z^+, \quad \forall j \\
& \sum_{i=1}^{n} v_i \cdot x_{ij} \geq z^-, \quad \forall j \\
& x_{ij} \in \{0, 1\}
\end{align*}

\textbf{Propiedad:} En optimalidad: $z^+ = \max_j V_j$ y $z^- = \min_j V_j$

\end{frame}

%=== SECCIÓN 3: COMPLEJIDAD ===
\section{Análisis de Complejidad}

\begin{frame}
\frametitle{Problema de Decisión vs Optimización}

\textbf{Problema de Decisión (BALANCED-BIN-PACKING-DEC):}
\begin{quote}
Dados $n$ ítems con pesos y valores, $k$ bins con capacidades $C_1, \ldots, C_k$, y un umbral $B$, ¿existe una asignación factible tal que la diferencia máxima de valores entre bins sea $\leq B$?
\end{quote}

\vspace{0.5cm}
\textbf{Problema de Optimización (BALANCED-BIN-PACKING-OPT):} 
\begin{quote}
Dados $n$ ítems con pesos y valores, $k$ bins con capacidades $C_1, \ldots, C_k$, encontrar una asignación factible que minimice la diferencia máxima de valores entre bins.
\end{quote}

\vspace{0.5cm}
\highlight{Proposición:} Si el problema de decisión BALANCED-BIN-PACKING-DEC está en NP, entonces el problema de optimización BALANCED-BIN-PACKING-OPT está en NPO (problemas de optimización NP).

\end{frame}

\begin{frame}
    \textbf{Paso 1: BALANCED-BIN-PACKING-DEC $\in$ NP}

Un certificado para una instancia con respuesta 'sí' es una asignación $\sigma: I \rightarrow \{1, \ldots, k\}$. La verificación requiere:
\begin{enumerate}
    \item Verificar que cada ítem está asignado: $O(n)$
    \item Calcular peso total de cada bin: $O(n)$
    \item Verificar restricciones de capacidad: $O(k)$
    \item Calcular valor total de cada bin: $O(n)$
    \item Verificar que $\max_j V_j - \min_j V_j \leq B$: $O(k)$
\end{enumerate}

Total: $O(n + k)$, por lo tanto el certificado es verificable en tiempo polinomial. $\square$

\end{frame}

\begin{frame}
\frametitle{Cadena de Reducciones}

\centering
\Large
\vspace{1cm}
PARTITION $\leq_p$ 3-PARTITION $\leq_p$ BALANCED-BIN-PACKING

\vspace{1cm}
\normalsize
\textbf{Implicación:} El problema es \textbf{NP-Hard}

\vspace{0.5cm}

\textbf{Resultado:} El problema es NP-Completo

\vspace{0.5cm}
No existe algoritmo polinomial conocido (asumiendo P $\neq$ NP)

\vspace{0.5cm}
Se requiere trade-off: \highlight{optimalidad vs velocidad}

\end{frame}

%=== SECCIÓN 4: MÉTODOS ===
\section{Métodos de Resolución}

\begin{frame}
\frametitle{Clasificación de Algoritmos}

\centering
\small
\begin{tabular}{|l|c|c|c|}
\toprule
\textbf{Método} & \textbf{Óptimo} & \textbf{Velocidad} & \textbf{Rango} \\
\midrule
\textbf{Exactos:} \\
Fuerza Bruta & Sí & Lenta & $n < 15$ \\
Branch \& Bound & Sí & Media & $n < 25$ \\
Prog. Dinámica & Sí & Media & $n < 20$ \\
\midrule
\textbf{Aproximados:} \\
Greedy (FFD/LPT) & No & Muy rápida & Todos \\
\midrule
\textbf{Metaheurísticas:} \\
Simulated Annealing & No & Rápida & Todos \\
Algoritmos Genéticos & No & Rápida & Todos \\
Búsqueda Tabú & No & Rápida & Todos \\
\bottomrule
\end{tabular}

\end{frame}

\begin{frame}
\frametitle{Algoritmos Greedy}

\textbf{First Fit Decreasing (FFD):}
\begin{enumerate}
    \item Ordenar ítems por peso (descendente)
    \item Asignar al primer contenedor con espacio
\end{enumerate}

\textbf{Complejidad:} $O(n \log n)$

\vspace{0.5cm}
\textbf{LPT Balanced:}
\begin{enumerate}
    \item Ordenar ítems por valor (descendente)
    \item Asignar al contenedor con menos carga
\end{enumerate}

\textbf{Complejidad:} $O(n \log n)$

\vspace{0.3cm}
\highlight{Ventajas:} Muy rápidos, buenos resultados

\highlight{Desventajas:} No garantizan optimalidad

\end{frame}

\begin{frame}
\frametitle{Fuerza Bruta}

Probar las $k^n$ asignaciones posibles.

\vspace{0.5cm}
\textbf{Límites prácticos:}
\begin{itemize}
    \item $k=2$: hasta $n=14$ (1 segundo)
    \item $k=3$: hasta $n=11$ (1 segundo)
    \item $k=4$: hasta $n=8$ (1 segundo)
\end{itemize}

\vspace{0.5cm}
\textbf{Complejidad:} $O(k^n \cdot n)$

\vspace{0.3cm}
\textbf{Rol:} Validar heurísticas, instancias pequeñas

\end{frame}

\begin{frame}
\frametitle{Branch and Bound}

\textbf{Idea:} Fuerza Bruta + poda inteligente

\vspace{0.3cm}
Explora árbol de decisiones calculando cotas inferiores.

\vspace{0.3cm}
Si cota $\geq$ mejor solución: \highlight{poda la rama}

\vspace{0.5cm}
\textbf{Ventajas:}
\begin{itemize}
    \item Garantiza optimalidad
    \item Mejora significativa en práctica
    \item Efectivo con cotas ajustadas
\end{itemize}

\vspace{0.3cm}
\textbf{Complejidad:} $O(k^n)$ peor caso

\end{frame}

\begin{frame}
\frametitle{Programación Dinámica}

\textbf{Idea:} Construir solución bin por bin.

\vspace{0.3cm}
\textbf{Estado:} $DP[j][mask]$ = mejor asignación de ítems en $mask$ a primeros $j$ contenedores

\vspace{0.3cm}
\textbf{Transición:} Probar subconjuntos del contenedor $j$

\vspace{0.5cm}
\textbf{Complejidad:}
\begin{itemize}
    \item Tiempo: $O(k^2 \cdot 3^n)$
    \item Espacio: $O(k \cdot 2^n)$
\end{itemize}

\textbf{Aplicabilidad:} Instancias pequeñas ($n < 20$)

\end{frame}

\begin{frame}
\frametitle{Metaheurísticas}

Simulated Annealing, Algoritmos Genéticos, Búsqueda Tabú

\vspace{0.5cm}
\textbf{Esquema general:}
\begin{enumerate}
    \item Generar solución inicial
    \item Mejorar iterativamente con movimientos locales
    \item Aceptar movimientos malos ocasionalmente
    \item Detener por convergencia o tiempo
\end{enumerate}

\vspace{0.5cm}
\textbf{Ventajas:} Rápidas, buenos resultados

\textbf{Desventajas:} Sin garantías, aleatoriedad, parámetros

\end{frame}

%=== SECCIÓN 5: ANÁLISIS ===
\section{Análisis Experimental}

\begin{frame}
\frametitle{Comparación de Tiempos}

\centering
\textbf{[GRÁFICO 1: Tiempo vs tamaño (escala log)]}

\vspace{2cm}
\small
Instancias aleatorias con $n \in [5, 50]$ e $k=3,4,5$

Eje Y: tiempo (segundos)

Eje X: número de ítems

\end{frame}

\begin{frame}
\frametitle{Órdenes de Complejidad}

\centering
\textbf{[GRÁFICO 2: Comparación de órdenes]}

\vspace{2cm}
\small
- FFD/LPT: línea plana (polinomial)

- B\&B: exponencial débil

- DP: exponencial fuerte $3^n$

- Metaheurísticas: línea plana (parámetro fijo)

\end{frame}

\begin{frame}
\frametitle{Calidad de Soluciones}

\textbf{Gap de Optimalidad:}
\[
\text{Gap}(\%) = \frac{\text{Heurístico} - \text{Óptimo}}{\text{Óptimo}} \times 100
\]

\vspace{0.5cm}
\textbf{Resultados promedio:}
\begin{itemize}
    \item FFD: 5-15\%
    \item LPT: 3-10\%
    \item Genéticos: 0.5-3\%
    \item Tabú: 0.2-1.5\%
\end{itemize}

\vspace{0.5cm}
\centering
\textbf{[GRÁFICO 3: Gap vs tamaño de instancia]}

\end{frame}

\begin{frame}
\frametitle{Resumen Comparativo}

\centering
\small
\begin{tabular}{|l|c|c|c|}
\toprule
\textbf{Algoritmo} & \textbf{Óptimo} & \textbf{Velocidad} & \textbf{Uso} \\
\midrule
Fuerza Bruta & Sí & No & Instancias pequeñas \\
B\&B & Sí & Regular & Instancias medianas \\
Prog. Dinámica & Sí & Regular & Instancias pequeñas \\
\midrule
FFD & No & Sí & Baseline rápido \\
LPT & No & Sí & Baseline rápido \\
\midrule
Genéticos & No & Sí & Problemas reales \\
Tabú & No & Sí & Problemas reales \\
\bottomrule
\end{tabular}

\end{frame}

%=== CONCLUSIONES ===
\section{Conclusiones}

\begin{frame}
\frametitle{Conclusiones}

\highlight{1. NP-Hard:} No existe algoritmo polinomial (bajo P $\neq$ NP)

\vspace{0.3cm}
\highlight{2. Trade-off:} Optimalidad vs velocidad

\vspace{0.3cm}
\textbf{Elección depende del contexto:}
\begin{itemize}
    \item Instancias pequeñas: B\&B o DP
    \item Problemas reales: Metaheurísticas
    \item Baseline rápido: Greedy
\end{itemize}

\vspace{0.3cm}
\highlight{3. Herramientas:} 9 algoritmos, dashboard interactivo, benchmarking

\vspace{0.3cm}
\highlight{4. Aplicaciones:} Logística, distribución de carga, computación distribuida

\end{frame}

%=== AGRADECIMIENTOS ===
\begin{frame}
\frametitle{Agradecimientos}

\vspace{2cm}
\centering
\Large
\highlight{Gracias por su atención}

\vspace{1.5cm}
\normalsize
\textbf{Preguntas y Discusión}

\vspace{1cm}
\small
Código disponible en GitHub

\end{frame}

\end{document}
